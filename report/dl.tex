\subsection{Introduction}

Difference constraints have the form
\[
x - y \leq c
\]
where $c \in \Rationals$ or $c \in \Integers$. We will denote the set of
constraints as $\phi$. Note that since $\leq$ is anti-symmetric, one can express
equality $x - y = c$ as two inequalities: $x - y \leq c$ and $c \leq x - y$,
where the latter is equivalent to $y -x \leq -c$. Difference Logic is quite
interesting from the point of view of SMT solver because constraints of that
form arise quite naturally in many situations and quite a few problems can be
encoded using them. Examples include arrays bound checking or job scheduling.
Furthermore, there are well-known and efficient algorithms that can check
satisfiability of such constraints in polynomial time. The current approach to
this problem is based on transforming the constraints into graph representation,
such that for every constraint of the form $x - y \leq c$ we have an edge $y
\rightarrow x$ with weight $c$. We will denote such a graph as $G_\phi$. The key
result that allows efficient decision procedures is that the constraints are not
satisfiable if and only if there is a negative cycle in such a graph
\cite{bib:algorithms}.


\subsection{Algorithms}

The basic algorithm used for satisfiability checking, in the context of SMT
solvers, is presented in \cite{bib:ms_dl}. As already mentioned, the problem is
reduced to determining whether there is a negative cycle in the graph
representing the constraint. The intuition behind this approach is as follows.
Let us have a cycle
$x \xrightarrow{c_1} \ldots \xrightarrow{c_n} x$ such that
$c_1 + \ldots + c_n < 0$. If we convert it back to the form of inequalities we
would arrive at:
\begin{align*}
& x - y_1 \leq c_1 \\
& y_1 - y_2 \leq c_2 \\
& \ldots \\
& y_{n-1} - x \leq c_n
\end{align*}
Now adding all the inequalities by sides will result in
\[
x - x \leq c_1 + \ldots + c_n \\
\]
since the intermediate variables cancel out. Simplifying we get
\[
0 \leq c_1 + \ldots + c_n < 0 \\
\]
which is clearly a contradiction, indicating that the constraints are not
satisfiable.

The usual approach to check if there is a negative cycle in the graph is to run
the Bellman-Ford algorithm on the graph. For reference the algorithm is
presented in Table \ref{tab:bellman_ford}. Since the algorithm is an SSSP
algorithm (Single-Source Shortest Paths) the first step is to add a new vertex
$v_s$ and for every other vertex $v$ in the graph an edge $v_s \rightarrow v$
with weight equal to zero. This can not introduce new negative cycles and simply
makes it possible to detect negative cycles using Bellman-Ford algorithm even if
the graph consists of a few disconnected subgraphs. The time complexity of the
algorithm is $\BigO(|V| |E|)$, where $V$ is the set of vertices and
$E$ the set of edges. However, here we will use $n$ to denote the number of
variables and $m$ for the number of constraints, therefore the satisfiability
check of a set of constraints can be performed in $\BigO((m + n) n)$ time
(the additional $n$ comes from the added edges with source in $v_s$, by
modifying the algorithm it is possible to achieve $\BigO(m n)$).
Moreover, note that once the Bellman-Ford algorithm returned true, the $\delta$
mapping gives us a potential solution to the set of constraints.

\begin{table}
\caption{The Bellman-Ford algorithm}
\label{tab:bellman_ford}
\textbf{Input:} source vertex $s$

\textbf{Output:} $\delta$ (maps each vertex to its shortest distance from the
  source vertex) and $p$ (parent pointer --- maps each vertex to its parent in
  the shortest path tree rooted in $s$)

\begin{enumerate}
\item Set $\delta(s) \leftarrow 0$, and
  for every $v \in V \setminus \{ s \}$ set $\delta(v) \leftarrow \infty$.
  For every $v \in V$ set $p(v) \leftarrow nil$.
%
\item For $i = 1$ to $|V| -1$ do
  \begin{enumerate}
  \item For each edge $(u, v) \in E$, if $\delta(v) > \delta(u) + w(u, v)$ then
    \begin{itemize}
    \item Set $\delta(v) \leftarrow \delta(u) + w(u, v)$
    \item Set $p(v) \leftarrow u$
    \end{itemize}
  \end{enumerate}
%
\item For each edge $(u, v) \in E$, if $\delta(v) > \delta(u) + w(u, v)$ then
  return false.
%
\item Return true.
\end{enumerate}
\end{table}

Once we discovered that there is a negative cycle in the graph, it is often
useful (e.g. to generate theory conflicts) to find the edges that cause it. This
can be achieved by using the parent pointers returned by the Bellman-Ford
algorithm. If we assume that in the second step of the algorithm it was vertex
$v$ such that $\delta(v) > \delta(u) + w(u, v)$
then we can use the pointers to go back in the path. There must be some vertex
that will be twice in the list and forms a negative weight cycle.

Finally it is useful to notice that for the above outlined procedures, it does
not really matter whether the domain is $\Rationals$ or $\Integers$.

\subsection{SVPI}

In the above description we have considered only constraints of the form
$x - y \leq c$ but it is easy to relax this requirement a bit and also allow
$x \leq c$ (i.e. SVPI --- Single Variable Per Inequality). This can be done
quite easily by simply creating a fresh vertex $v_0$ and treating rewriting all
such constraints as $x - v_0 \leq c$. The above algorithm for satisfiability
checking will then work as expected. The only other thing to remember is that
$\delta$ returned by Bellman-Ford algorithm might have to be shifted in order to
get that $\delta(v_0) = 0$. This relies on the property that for any solution
$\delta$ to the difference constraints $\delta'(v) = \delta(v) + k$ is also a
solution (for every $k$). Intuitively, since we are only interested in the
differences between the variables, we can always "shift" the solution.

\subsection{Strict inequalities}

So far we have limited ourselves only to non-strict inequalities and have not
considered strict ones (i.e. $<$ instead of $\leq$) . However, the problem with
strict inequalities can be reduced into one with non-strict ones. In case of
$\Integers$ this is quite trivial ($x - y < 5$ is equivalent to $x - y \leq 4$).

But it is also possible for $\Rationals$ as presented in
\cite{bib:arithmetic_dpllt}. The solution is to use infinitesimal value
$\epsilon$ and transform the strict inequalities to the non-strict ones. For
instance consider $x - y < c$, we can just rewrite that as $x - y \leq c -
\epsilon$. The only necessary change is to slightly modify our domain (constants
and variable assignments). We need to define $\Rationals_\epsilon$ to be pairs of
rationals such that $(c, k)$ denotes $c + k \delta$. Finally we need to define a
few operations for $\Rationals_\epsilon$:
\begin{align*}
(c_1, k_1) + (c_2, k_2) & \equiv (c_1 + c_2, k_1 + k_2) \\
a \times (c, k) & \equiv (a \times c, a \times k) \\
(c_1, k_1) \leq (c_2, k_2) & \equiv (c_1 \leq c_2) \vee
                                    (c_1 = c_2 \land k_1 \leq k_2) \\
\end{align*}
Note that the last line above defines nothing else than standard lexicographical
ordering.


\subsection{Beyond satisfiability}

There are some possible improvements to the above procedure. One of them is to
use a different algorithm for detecting the negative cycle. There have been some
newer algorithms with slightly better performance than the Bellman-Ford
algorithm, see \cite{bib:negative_cycles} for a survey. For instance
\cite{bib:dl_propagation} uses an algorithm based on
\cite{bib:dynamic_negative_cycle}. Unfortunately since the experiments were
focused on theory propagation, it is hard to say how much this aspect can
influence the overall performance.

\begin{table}[b]
\caption{Algorithm for equality generation}
\label{tab:eq_gen}
\begin{enumerate}
\item Let $E'$ be the set ef edges in the graph such that $e \in E'$ if and only
  if $sl(e) = 0$
\item Create an induced subgraph $G'_\phi(V, E')$ from $G'_\phi(V, E)$.
\item Calculate strongly connected components (SCC) \cite{bib:tarjan_dfs} of the
  graph.
\item For each SCC S, let $V_d^s = \{ x \mid x \in S, \delta(x) = d \}$.
\item For each $V_d^s = \{ x_1, \ldots, x_k \}$ we have the equalities
  $x_1 = x_2 = \ldots = x_k$.
\end{enumerate}
\end{table}

Apart from that, as already mentioned, SMT solvers rarely work with constraints
of just one type and often employ a combination of various theories in the
Nelson-Oppen framework \cite{bib:nelson_oppen}. However, one of the requirements
of this framework is that all the theory solvers must share the implied
equalities. Therefore it is important to have an efficient algorithm also for
generating equalities that are implied by the given set of difference
constraints. Here we will briefly cover the one presented in \cite{bib:ms_dl}.
Of course we assume here that the set of constraint is satisfiable (i.e.
Bellman-Ford algorithm returned true for the generated graph). A naive way to
generate all the equalities is to perform the transitive closure of all the
constraints and check if we can find $x \leq y$ and $y \leq x$ for some $x$ and
$y$. However, this is quite expensive --- in the worst case would require
$\BigO(n^3)$. A more efficient algorithm is presented in Table \ref{tab:eq_gen}.
However, before we discuss it we need to introduce the concept of slack, i.e.
for every edge $(u, v)$ its slack is defined as
\[
sl(u, v) = \delta(v) - \delta(u) + w(u, v)
\]

The algorithm is based on the idea that if we consider only the edges have slack
equal to zero then any cycle formed by these edges will have weight equal to
zero. So when we have two vertices $u$ and $v$ in the same SCC (i.e. there is a
path from $u$ to $v$ and the other way around) and they have the same value
associated by $\delta$ then we have that the weight of both paths is equal to
zero. Which corresponds to $v - u \leq 0$ and $u - v \leq 0$, i.e. $u = v$.
