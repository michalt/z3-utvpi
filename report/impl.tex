\subsection{Introduction}

The accompanying UTPVI theory solver for Z3 SMT Solver \cite{bib:z3} is
implemented in around 1500 lines of C++ (taking advantage of recent C++11
standard \cite{bib:cc11}). It uses Boost Graph Library
\cite{bib:boost,bib:boost_graph} for graph manipulation and GMP \cite{bib:gmp}
for arbitrary precision arithmetic. Due to time constraints it is mainly a
proof-of-concept and does not have all the features necessary for usage in a
combination with other theories. Currently it implements:
\begin{itemize}
\item decision procedure for UTVPI (and thus SVPI) for integers, both using
  basic types of C++ like int, as well as arbitrary precision integers from GMP
  library
\item decision procedure for UTVPI (and thus SVPI) for rationals using arbitrary
  precision rationals from GMP library\footnote{In fact current implementation
  should work with basic types like \texttt{float} or \texttt{double}, however,
  due to their imprecision, this would probably not be very useful.}
\item generation of theory conflicts
\item backtracking
\item integration with Z3
\end{itemize}
Initially the implementation was to be based on the one for Fx7 SMT solver
\cite{bib:fx7}. However, it implements an older algorithm described in
\cite{bib:harvey_stuckey}. Therefore this implementation is not based on the
procedure from Fx7 and implements the more recent algorithms from
\cite{bib:ms_utvpi} instead.

One of the main highlights of this implementation is the ability to easily
instantiate solvers for rationals/integers with various representations. For
instance the following snippet creates theories for checking satisfiability for
integers that fit in \texttt{int}, integers of arbitrary precision and rationals
of arbitrary precision respectively.
\begin{lstlisting}
Z3_theory mach_integers  = MkTheory<UtvpiGraphZ, int>(context);
Z3_theory arbt_integers  = MkTheory<UtvpiGraphZ, mpz_class>(context);
Z3_theory arbt_rationals = MkTheory<UtvpiGraphQ, mpq_class>(context);
\end{lstlisting}
Where \texttt{context} is \texttt{Z3\_context}, \texttt{mpz\_class} and
\texttt{mpq\_class} are GMP's classes for arbitrary precision integers and
rationals respectively\footnote{GMP offers both a basic C as well as somewhat
more convenient C++ interface, which overloads basic arithmetic operators.}. Note
that this includes running all the described graph algorithms with weights of
the specified types, as well as using the appropriate functions from Z3 to parse
and create numerals (in case of \texttt{int} we use
\texttt{Z3\_get\_numeral\_int}, in case of \texttt{mpz\_class} we need to
go through a string and use
use \texttt{Z3\_get\_numeral\_string}).

The project uses Z3 parser and predefines two predicates \texttt{Utvpi} and
\texttt{Svpi} and introduces additional sort \texttt{Sign} with two constants
\texttt{Minus} and \texttt{Plus} that can be used as follows:
\begin{verbatim}
(declare-fun x () Int)
(declare-fun y () Int)

(assert (or (Utvpi Plus x Minus y (~ 1)) (Utvpi Plus x Minus y 1)))
(assert (Svpi Plus y 0))
(assert (Svpi Minus x (~ 1)))
\end{verbatim}
This translates to
\[
(x - y \leq -1 \vee x - y \leq 1 )
\land
y \leq 0
\land
- x \leq -1
\]
Of course this can be easily changed as desired.

The tool also outputs some information on how it is solving the problem. For the
above example the output is:
\begin{verbatim}
Parsing file: test.smt
Parsed the following formula:
(and (or (Utvpi Plus x Minus y (- 1)) (Utvpi Plus x Minus y 1))
     (Svpi Plus y 0)
     (Svpi Minus x (- 1)))
Z3: Assigned (Svpi Plus y 0) to 1
Z3: Assigned (Svpi Minus x -1) to 1
Z3: Push
Z3: Assigned (Utvpi Plus x Minus y -1) to 1
UtvpiGraphZ: Found negative weight cycle without tightening.
UtvpiGraph: Reason for negative cycle:
[..]
SatCheck: UNSAT.
SatCheck: asserting theory axiom:
(not (and (and (and (Svpi Minus x -1) (Utvpi Plus x Minus y -1))
               (Svpi Plus y 0))
          (Utvpi Plus x Minus y -1)))
Z3: Pop
Z3: Assigned (Utvpi Plus x Minus y 1) to 1
Z3: SAT
Z3: Reset
\end{verbatim}
Which clearly shows how everything works:
\begin{enumerate}
\item First Z3 assigns the two \texttt{SVPI} predicates the value true (they
  are asserted so they must be true for the formula to by satisfiable).
\item Then Z3 makes \texttt{Push} callback for the case-split caused by the
  asserted disjunction.
\item Assigning true for the first disjuncts causes a theory conflict from the
  UTVPI solver (clearly $y \leq 0 \land x \geq 1 \land x - y \leq -1$ is
  unsatisfiable).
\item Z3 reacts making the \texttt{Pop} callback and assigning true to the other
  disjunct.
\item Theory solver returns that this is satisfiable.
\item Thus the formula is satisfiable.
\end{enumerate}
Of course if we modify the formula removing the second disjunct, the result is:
\begin{verbatim}
Parsing file: test.smt
Parsed the following formula:
(and (Utvpi Plus x Minus y (- 1)) (Svpi Plus y 0) (Svpi Minus x (- 1)))
Z3: Assigned (Utvpi Plus x Minus y -1) to 1
Z3: Assigned (Svpi Plus y 0) to 1
Z3: Assigned (Svpi Minus x -1) to 1
UtvpiGraphZ: Found negative weight cycle without tightening.
UtvpiGraph: Reason for negative cycle:
[..]
SatCheck: UNSAT.
SatCheck: asserting theory axiom:
(not (and (and (and (Utvpi Plus x Minus y -1) (Svpi Plus y 0))
               (Utvpi Plus x Minus y -1))
          (Svpi Minus x -1)))
Z3: UNSAT
Z3: Reset
\end{verbatim}
