\subsection{Introduction and comparison with DL}

UTVPI stands for Unit Two Variable Per Inequality and is also referred to as
Octagons \cite{bib:octagons}. Here we will be interested mostly in the question
of satisfiability of a set of constraints and not in the details of the
constraints as an abstract domain. In general the UTVPI constraints are of the
form
\[
a x + b y \leq c
\]
where $a, b \in \{ -1, 0, 1 \}$ and $c$ is either $\Rationals$ or $\Integers$.
UTVPI constraints are quite interesting because they are a bit more expressive
than difference logic and so allow more problems to be encoded, yet there still
exist efficient polynomial decision procedures for them. Furthermore in case of
$\Integers$ relaxing any requirement of the UTVPI (allowing non-unit
coefficients or more variables) leads to NP-complete decision procedures.

The first decision procedure for UTVPI was presented in \cite{bib:beyond_finite}
and showed that transitive closure of the set of constraints with respect to the
transitive and in case of $\Integers$ also tightening rules (Table
\ref{tab:inference}) is unsatisfiable if and only if it contains a
contradiction, like $0 \leq d$ where $d < 0$.

Note the importance of the tightening rule for $\Integers$. The additional
difficulty of $\Integers$ constraints is that they can have solutions that are
not in $\Integers$. So they might be satisfiable in $\Rationals$ but not in
$\Integers$. The tightening rule basically provides additional information that
if we have a constraint that is equivalent to $2x \leq d$ and $d$ is odd then
we can tighten the bound on $x$ (i.e. we have $2x \leq d - 1$).

\begin{table}[b]
\caption{Inference rules for UTVPI}
\label{tab:inference}
\centering
\begin{align}
& \frac{ ax + by \leq c  \qquad  -ax + b'z \leq d }
     { by +b'z \leq c + d}
\tag{Transitive} \\
& \notag \\
& \frac{ ax + by \leq c  \qquad  ax - by \leq d }
     { ax \leq \lfloor (c + d) / 2 \rfloor }
\tag{Tightening}
\end{align}
\end{table}
%
Also note that we will not consider here the case of strict inequalities ---
they can be handled as in case of DL. Instead we will focus more on the case
when the domain is $\Integers$, because for UTVPI this case is actually
more difficult than the one with $\Rationals$.

\subsection{Algorithms}

The procedure presented in \cite{bib:beyond_finite} has been improved by
\cite{bib:harvey_stuckey} but the worst case time complexity remained the same
--- $\BigO(m n^2)$. However, a new approach was proposed in \cite{bib:ms_utvpi}
which does not involve the transitive closure and uses a graph representation of
the constraints to solve satisfiability (note the similarity to the difference
logic). The complexity of the procedure is equal to the complexity of negative
cycle detection algorithm. So using Bellman-Ford algorithm it is possible to
achieve $\BigO(m n)$. One of the main ideas behind the approach is to construct
a graph with two vertices for each variable.

For every variable $x$ in the set of constraints we have two vertices $x^-$ and
$x^+$ that represent $-x$ and $x$. We will sometimes use $-v$ to denote $x^-$ if
$v$ represents $x^+$ and the other way around. Now for every constraint we add
one or two edges, as presented in the Table \ref{tab:utvpi_edges}.

\begin{table}
\caption{Edges in the UTVPI graph}
\label{tab:utvpi_edges}
\[
\begin{array}{c|c|c}
\textrm{UTVPI constraint} & \textrm{DL constraints} & \textrm{Graph Edges} \\
\hline \hline
x - y \leq c &
  x^+ - y^+ \leq c        \qquad y^- - x^- \leq c &
  y^+ \xrightarrow{c} x^+ \qquad x^- \xrightarrow{c} y^- \\
\hline
x + y \leq c &
  x^+ - y^- \leq c        \qquad y^+ - x^- \leq c &
  y^- \xrightarrow{c} x^+ \qquad x^- \xrightarrow{c} y^+ \\
\hline
- x - y \leq c &
  x^- - y^+ \leq c        \qquad y^- - x^+ \leq c &
  y^+ \xrightarrow{c} x^- \qquad x^+ \xrightarrow{c} y^- \\
\hline
x \leq c &
  x^+ - x^- \leq c &
  x^- \xrightarrow{2c} x^+ \\
\hline
-x \leq c &
  x^- - x^+ \leq c &
  x^+ \xrightarrow{2c} x^- \\
\end{array}
\]
\end{table}

There are a few interesting properties of such a graph. First of all the graph
is very similar to the one from decision procedure for DL. Yet it does not need
a special vertex $v_0$ that was is needed it in the constraint graphs for
difference logic to handle constraints of the from $x \leq c$. Another important
observation about this new graph is that whenever there is an edge $(u, v)$
there also is an edge $(-v, -u)$. Moreover this can be extended to paths ---
whenever we have a path from $u$ to $v$, there is a path from $-v$ to $-u$.
Apart from that, let $\delta$ be a valuation of all the vertices (i.e. a possible
solution to the constraints), then whenever we have a shortest path $SP(u, v)$ between
two vertices $u$ and $v$, we know that $\delta(v) - \delta(u) \leq wSP(u,
v)$ (where $wSP$ stands for $w \circ SP$, that is the weight of the shortest
path). Finally, one of the main result with respect to UTVPI/Octagons is the
following \cite{bib:octagons}.
%
\begin{lemma}
A set of UTVPI constraints is unsatisfiable in $\Rationals$ if and only if the
corresponding constraint graph contains a negative weight cycle.
\end{lemma}
%
Note that this is only true for $\Rationals$ and not for $\Integers$. As already
mentioned constraints over integers pose additional difficulties since they can
have a non-integer solutions. Therefore for the case of integers we need to
introduce the concept of tightening edges. They are defined as follows:
\[
T = \{ (u, -u) \mid wSP(u, -u) \text{ is odd} \}
\]
Note that this corresponds to an either $u^+ - u^- \leq c$ or $u^- - u^+ \leq
c$, where $c$ is odd. We are interested only in those edges because these are the
only ones that can cause the Tightening rule to apply. Therefore for each edge
in $T$ we define its weight as follows
\[
w_T(u, -u) = wSP(u, -u) - 1
\]
Let us denote the graph with those additional edges as $G_{\phi \cup T}$. One of
the main contributions of \cite{bib:ms_utvpi} is the lemma below.
\begin{lemma}
A set of UTVPI constraint $\phi$ is unsatisfiable in $\Integers$ if and only if
the graph $G_{\phi \cup T}$ has a negative weight cycle.
\end{lemma}
The na\"{i}ve approach would be to actually find and add all the additional
edges and run the negative cycle detection algorithm on this new graph. This
could be achieved by using either Floyd-Warshall or Johnson's All Pairs Shortest
Paths algorithms to find the tightening edges and then use negative cycle
detection algorithm. However, \cite{bib:ms_utvpi} presents a bit smarter
approach. It is summarized in Tabel \ref{tab:utvpi_sat}.

\begin{table}
\caption{Algorithm for satisfiability of UTVPI constraints}
\label{tab:utvpi_sat}
\begin{enumerate}
\item Construct $G_\phi$ graph.
\item Run a negative cycle detection algorithm on the graph.
  \begin{enumerate}
  \item If there is a negative cycle, return UNSAT.
  \item Otherwise the algorithm will return shortest paths from the source
    vertex $v_s$ to all other vertices. This can be treated as a feasible solution
    to the constraints. Note that $\delta(v) - \delta(u) \leq w(u, v)$ for all
    edges $(u, v)$.
  \item Let $E'$ be the set of edges from $G$ such that $(u, v) \in E'$ iff
    $\delta(u) - \delta(v) = w(u, v)$.
  \item Create graph $G'_\phi$ induced by $E'$
  \item Group the vertices into strongly connected components (for instance
    using Tarjan's algorithm \cite{bib:tarjan_dfs}, which is
    $\BigO(|V| + |E|)$).
  \item For each vertex $u \in V$, if $-u$ is in the same SCC as $u$ and
    $\delta(u) - \delta(-u)$ is odd then return UNSAT.
  \item Return SAT.
  \end{enumerate}
\end{enumerate}
\end{table}
The intuition behind the second part of the algorithm (i.e. SCC computation) is
to identify zero weight cycles and check if we can tighten some edge. If so then
the zero cycle in $G_\phi$ will result in a negative one in $G_{\phi \cup T}$ (i.e.
after adding the tightening edge). Moreover whenever there is a path $P$ between
$u$ and $v$ such that all the edges along the path have slack equal to zero,
then $wSP(u, v) = \delta(v) - \delta(u)$. Now if $u$ and $-u$ are in the same
SCC, then there is a path from $u$ to $-u$ and from $-u$ to $u$.
Moreover
\begin{align*}
\delta(u) - \delta(-u) & = wSP(-u, u) \\
\delta(-u) - \delta(u) & = wSP(u, -u) \\
0 & = wSP(-u, u) + wSP(u, -u) \\
\end{align*}
So $u$ and $-u$ form a zero weight cycle. If one of the paths is odd, then we
know that a corresponding tightening edge will be in $T$ and so from a negative
cycle. For more precise description and the formal soundness and completeness
proofs we refer to \cite{bib:ms_utvpi}.

The solver to be useful in DPLL(T) should produce theory conflicts, in this case
which inequalities cause a negative cycle. Then the solver can assert the theory
axiom saying that at least one of those equalities must be false. In order to
find the inequalities, we must track the reason why a given edge was added to
the graph. Then once we detect a negative cycle using e.g. Bellman-Ford
algorithm we can simply use the parent pointers to find the cycle. If the
unsatisfiability was detected in the second part of the algorithm, then all we
need to do is to find the path from $u$ to $-u$ and back in the induced graph
(this can be done by Breadth First Search) --- again the cycle is a proof that
there are no solutions in $\Integers$.

Finally, as in the case of DL, in order for the theory solver to be used in the
DPLL(T) framework, should be capable of generating all equalities that are
consequences of the considered inequalities. We will cover here two possible
approaches, as introduced in \cite{bib:ms_utvpi}. The first, quite simple way of
generating all inequalities is presented in Table~\ref{tab:naive_eq}. The idea
behind the algorithm is pretty much the same as for the one for difference
logic. That is if we know that
$\delta_{\phi \cup T}(x^+) = \delta_{\phi \cup T}(y^+)$
and both vertices are in a cycle of weight zero, then they are equal.

\begin{table}
\caption{Simple algorithm for equality generation}
\label{tab:naive_eq}
\begin{enumerate}
\item Construct $G_{\phi \cup T}$ graph and run a negative cycle detection
  algorithm to calculate a feasible solution $\delta_{\phi \cup T}$.
\item Let $E_0$ be a set of edges such that $e \in E_0$ iff $sl(e) = 0$.
\item Create the subgraph $G_0$ induced by the $E_0$ set and group the vertices
  into SCCs.
\item If vertices $x^+$ and $y^+$ are in the same SCC and
  $\delta_{\phi \cup T}(x^+) = \delta_{\phi \cup T}(y^+)$, then report that
  $x = y$.
\end{enumerate}
\end{table}

\begin{table}
\caption{Improved algorithm for equality generation}
\label{tab:fast_eq}
\begin{enumerate}
\item Construct graph $G_\phi$ and run the satisfiability algorithm on it. We
  assume that the algorithm returned SAT and a possible solution $\delta$
\item Calculate $E_2 = \{ (u, v) \mid sl(u, v) \leq 2 \}$ and the graph $G_2$
  induced by $E_2$.
\item Let $T_2$ be the set of tightening edges (initially set to $\emptyset$)
  that we are interested in.
\item For each vertex $v$
  \begin{enumerate}
  \item Find the path $P_v$ in $G_2$ from $v$ to $-v$ with the smallest slack (if
    any). This can be achieved using breadth-first search or Dijkstra's
    algorithm.
  \item If $P_v$ exists
    \begin{enumerate}
    \item Let $wSP(v, -v) = \delta(-v) - \delta(v) + sl(P_v)$.
    \item If $wSP(v, -v)$ is odd then add the edge $(v, -v)$ to $T_2$ and assign
      it weight $w_{T_2}(v, -v) = wSP(v, -v) - 1$.
    \end{enumerate}
  \end{enumerate}
\item Add all the edge in $T_2$ to the $G_\phi$ graph to get
  $G_{\phi \cup T_2}$.
\item Proceed as in the previous algorithm but with
  $G_{\phi \cup T_2}$ instead of $G_{\phi \cup T}$.
\end{enumerate}
\end{table}
However, this way of calculating the equalities requires us to compute all the
tightening edges $T$ and then add them to the graph. As before an improved
algorithm would avoid finding all those edges and use only the ones that can
lead to a cycle of weight zero. The improved algorithm makes use of an
intermediate lemma that is presented below \cite{bib:ms_utvpi}.

\begin{lemma}
Assuming $G_{\phi \cup T}$ has no negative cycles and if $C$ is a zero weight
cycle in $G_{\phi \cup T}$ containing a tightening edge $(u, -u)$ then there is
a cycle $C'$ in $G_\phi$ containing $u$ and $-u$ such that $w(C') \leq 2$.
\end{lemma}

This builds on another property that whenever there is a cycle $C$ in $G_{\phi \cup
T}$ then there is cycle $C'$ with at most two tightening edges, such that $w(C')
< 0$ or $w(C') < w(C)$. This gives us the ability to consider, without loss of
generality, cycles with at most two tightening edges.

Now the idea behind the finding only the tightening edges that are useful for
equality generation is that if the cycle $C$ has at most two tightening edges
then there is a cycle $C'$ without those edges and its weight is not greater
than two (i.e. each of the tightening edges can decrease the weight by one only,
so if we remove them the weight can get larger by at most two).
The improved algorithm that taking advantage of this is presented in the Table
\ref{tab:fast_eq}. It only really changes the first step of the previous one ---
instead of calculating the whole $G_{\phi \cup T}$, it calculates only those
tightening edges that are useful, adds them to the $G_\phi$ graph and then
proceeds as before.
