\documentclass[a4paper,10pt]{article}

\usepackage[utf8]{inputenc}
\usepackage[T1]{fontenc}
\usepackage{lmodern}
\usepackage[pdftex]{hyperref}
\usepackage[pdftex]{graphicx}
\usepackage{listings}
\usepackage{amsmath}
\usepackage{amssymb}
\usepackage{enumerate}
\usepackage{color}
\usepackage{verbatim}

\usepackage{tikz}
\usetikzlibrary{shapes,arrows}

\tikzstyle{circ} = [ circle, draw, minimum height=2em ]
\tikzstyle{line} = [draw, -latex']
\tikzstyle{none} = []

\lstset{ %
  language=C++,
  basicstyle=\footnotesize,
  numbers=left,
  numberstyle=\footnotesize,
  stepnumber=1,
}
\newtheorem{proposition}{Proposition}
\newtheorem{lemma}{Lemma}

\newcommand{\Domain}{\mathbb{D}}
\newcommand{\Integers}{\mathbb{Z}}
\newcommand{\Rationals}{\mathbb{Q}}
\newcommand{\Reals}{\mathbb{R}}
\newcommand{\BigO}{\mathcal{O}}

\begin{document}

\pagestyle{headings}

\title{Advanced Analysis Techniques \\ SMT Solvers}
\author{Micha\l\ Terepeta}

\date{January, 2012}

\maketitle

{\abstract The aim of this report is to provide a brief survey of current
approaches to satisfiability of Difference Logic (DL) and Unit Two Variable Per
Inequality (UTVPI) constraints. First the report covers DL with the basic
algorithms and complexity results, then goes on to describe the UTVPI case with
recent algorithms. In both cases satisfiability as well as equality generation
is covered. Furthermore the report is accompanied by a proof-of-concept
implementation of UTVPI theory solver for the Z3 SMT.}


\tableofcontents

\newpage
\section{Introduction}

In many software verification projects, it is necessary to check for
satisfiability of certain constraints. However, propositional satisfiability,
is often unsuitable for direct use in cases that involve e.g. arithmetic
constraints. This is one of the reasons for growing interest in SMT
(Satisfiability Modulo Theories). SMT solvers provide a much richer
language (or actually languages) that one can use to communicate the
constraints. The commonly available theories include free/uninterpreted
functions, equality, arrays, difference constraints, linear arithmetic, etc.

Many of these theories naturally arise in software verification, however, they
rarely occur just by themselves --- more often the constraints are defined over a
combination of various theories. This leads us to one of the main achievements
in SMT --- the Nelson-Oppen framework \cite{bib:nelson_oppen} that specifies how
one can combine various theories and still get a decidable algorithm for
satisfiability. Therefore modern SMT solvers are capable of solving the problem
of satisfiability of constraints over many different theories. This is one of
the reasons of why more and more static analyzers/software model checkers use
SMT solvers to prove (or disprove) various properties about analyzed software.

Finally there is, as always in case of software verification, the question of
performance. Today's SMT solvers use many very specialized algorithms for
various theories. As an example consider the fact that many SMT solvers have a
separate engine for linear arithmetic and for difference logic (even though
difference logic is a special case of linear arithmetic). The reason for that is
simple --- one of the best algorithms for linear real arithmetic called Simplex
is exponential in the worst case, whereas the problem of satisfiability of
linear integer arithmetic is NP-complete. On the other hand, difference
constraints can be solved by using very efficient graph-based algorithms in
polynomial time. Therefore specialization can often be quite beneficial. This
report is exactly about exploring a specialized solver for UTVPI constraints,
which are a slightly more expressive fragment of linear arithmetic yet still
allows efficient polynomial satisfiability algorithms.

%
% Difference Logic
%

\newpage
\section{Difference Logic}
\subsection{Introduction}

Difference constraints have the form
\[
x - y \leq c
\]
where $c \in \Rationals$ or $c \in \Integers$. We will denote the set of
constraints as $\phi$. Note that since $\leq$ is anti-symmetric, one can express
equality $x - y = c$ as two inequalities: $x - y \leq c$ and $c \leq x - y$,
where the latter is equivalent to $y -x \leq -c$. Difference Logic is quite
interesting from the point of view of SMT solver because constraints of that
form arise quite naturally in many situations and quite a few problems can be
encoded using them. Examples include arrays bound checking or job scheduling.
Furthermore, there are well-known and efficient algorithms that can check
satisfiability of such constraints in polynomial time. The current approach to
this problem is based on transforming the constraints into graph representation,
such that for every constraint of the form $x - y \leq c$ we have an edge $y
\rightarrow x$ with weight $c$. We will denote such a graph as $G_\phi$. The key
result that allows efficient decision procedures is that the constraints are not
satisfiable if and only if there is a negative cycle in such a graph
\cite{bib:algorithms}.


\subsection{Algorithms}

The basic algorithm used for satisfiability checking, in the context of SMT
solvers, is presented in \cite{bib:ms_dl}. As already mentioned, the problem is
reduced to determining whether there is a negative cycle in the graph
representing the constraint. The intuition behind this approach is as follows.
Let us have a cycle
$x \xrightarrow{c_1} \ldots \xrightarrow{c_n} x$ such that
$c_1 + \ldots + c_n < 0$. If we convert it back to the form of inequalities we
would arrive at:
\begin{align*}
& x - y_1 \leq c_1 \\
& y_1 - y_2 \leq c_2 \\
& \ldots \\
& y_{n-1} - x \leq c_n
\end{align*}
Now adding all the inequalities by sides will result in
\[
x - x \leq c_1 + \ldots + c_n \\
\]
since the intermediate variables cancel out. Simplifying we get
\[
0 \leq c_1 + \ldots + c_n < 0 \\
\]
which is clearly a contradiction, indicating that the constraints are not
satisfiable.

The usual approach to check if there is a negative cycle in the graph is to run
the Bellman-Ford algorithm on the graph. For reference the algorithm is
presented in Table \ref{tab:bellman_ford}. Since the algorithm is an SSSP
algorithm (Single-Source Shortest Paths) the first step is to add a new vertex
$v_s$ and for every other vertex $v$ in the graph an edge $v_s \rightarrow v$
with weight equal to zero. This can not introduce new negative cycles and simply
makes it possible to detect negative cycles using Bellman-Ford algorithm even if
the graph consists of a few disconnected subgraphs. The time complexity of the
algorithm is $\BigO(|V| |E|)$, where $V$ is the set of vertices and
$E$ the set of edges. However, here we will use $n$ to denote the number of
variables and $m$ for the number of constraints, therefore the satisfiability
check of a set of constraints can be performed in $\BigO((m + n) n)$ time
(the additional $n$ comes from the added edges with source in $v_s$, by
modifying the algorithm it is possible to achieve $\BigO(m n)$).
Moreover, note that once the Bellman-Ford algorithm returned true, the $\delta$
mapping gives us a potential solution to the set of constraints.

\begin{table}
\caption{The Bellman-Ford algorithm}
\label{tab:bellman_ford}
\textbf{Input:} source vertex $s$

\textbf{Output:} $\delta$ (maps each vertex to its shortest distance from the
  source vertex) and $p$ (parent pointer --- maps each vertex to its parent in
  the shortest path tree rooted in $s$)

\begin{enumerate}
\item Set $\delta(s) \leftarrow 0$, and
  for every $v \in V \setminus \{ s \}$ set $\delta(v) \leftarrow \infty$.
  For every $v \in V$ set $p(v) \leftarrow nil$.
%
\item For $i = 1$ to $|V| -1$ do
  \begin{enumerate}
  \item For each edge $(u, v) \in E$, if $\delta(v) > \delta(u) + w(u, v)$ then
    \begin{itemize}
    \item Set $\delta(v) \leftarrow \delta(u) + w(u, v)$
    \item Set $p(v) \leftarrow u$
    \end{itemize}
  \end{enumerate}
%
\item For each edge $(u, v) \in E$, if $\delta(v) > \delta(u) + w(u, v)$ then
  return false.
%
\item Return true.
\end{enumerate}
\end{table}

Once we discovered that there is a negative cycle in the graph, it is often
useful (e.g. to generate theory conflicts) to find the edges that cause it. This
can be achieved by using the parent pointers returned by the Bellman-Ford
algorithm. If we assume that in the second step of the algorithm it was vertex
$v$ such that $\delta(v) > \delta(u) + w(u, v)$
then we can use the pointers to go back in the path. There must be some vertex
that will be twice in the list and forms a negative weight cycle.

Finally it is useful to notice that for the above outlined procedures, it does
not really matter whether the domain is $\Rationals$ or $\Integers$.

\subsection{SVPI}

In the above description we have considered only constraints of the form
$x - y \leq c$ but it is easy to relax this requirement a bit and also allow
$x \leq c$ (i.e. SVPI --- Single Variable Per Inequality). This can be done
quite easily by simply creating a fresh vertex $v_0$ and treating rewriting all
such constraints as $x - v_0 \leq c$. The above algorithm for satisfiability
checking will then work as expected. The only other thing to remember is that
$\delta$ returned by Bellman-Ford algorithm might have to be shifted in order to
get that $\delta(v_0) = 0$. This relies on the property that for any solution
$\delta$ to the difference constraints $\delta'(v) = \delta(v) + k$ is also a
solution (for every $k$). Intuitively, since we are only interested in the
differences between the variables, we can always "shift" the solution.

\subsection{Strict inequalities}

So far we have limited ourselves only to non-strict inequalities and have not
considered strict ones (i.e. $<$ instead of $\leq$) . However, the problem with
strict inequalities can be reduced into one with non-strict ones. In case of
$\Integers$ this is quite trivial ($x - y < 5$ is equivalent to $x - y \leq 4$).

But it is also possible for $\Rationals$ as presented in
\cite{bib:arithmetic_dpllt}. The solution is to use infinitesimal value
$\epsilon$ and transform the strict inequalities to the non-strict ones. For
instance consider $x - y < c$, we can just rewrite that as $x - y \leq c -
\epsilon$. The only necessary change is to slightly modify our domain (constants
and variable assignments). We need to define $\Rationals_\epsilon$ to be pairs of
rationals such that $(c, k)$ denotes $c + k \delta$. Finally we need to define a
few operations for $\Rationals_\epsilon$:
\begin{align*}
(c_1, k_1) + (c_2, k_2) & \equiv (c_1 + c_2, k_1 + k_2) \\
a \times (c, k) & \equiv (a \times c, a \times k) \\
(c_1, k_1) \leq (c_2, k_2) & \equiv (c_1 \leq c_2) \vee
                                    (c_1 = c_2 \land k_1 \leq k_2) \\
\end{align*}
Note that the last line above defines nothing else than standard lexicographical
ordering.


\subsection{Beyond satisfiability}

There are some possible improvements to the above procedure. One of them is to
use a different algorithm for detecting the negative cycle. There have been some
newer algorithms with slightly better performance than the Bellman-Ford
algorithm, see \cite{bib:negative_cycles} for a survey. For instance
\cite{bib:dl_propagation} uses an algorithm based on
\cite{bib:dynamic_negative_cycle}. Unfortunately since the experiments were
focused on theory propagation, it is hard to say how much this aspect can
influence the overall performance.

\begin{table}[b]
\caption{Algorithm for equality generation}
\label{tab:eq_gen}
\begin{enumerate}
\item Let $E'$ be the set ef edges in the graph such that $e \in E'$ if and only
  if $sl(e) = 0$
\item Create an induced subgraph $G'_\phi(V, E')$ from $G'_\phi(V, E)$.
\item Calculate strongly connected components (SCC) \cite{bib:tarjan_dfs} of the
  graph.
\item For each SCC S, let $V_d^s = \{ x \mid x \in S, \delta(x) = d \}$.
\item For each $V_d^s = \{ x_1, \ldots, x_k \}$ we have the equalities
  $x_1 = x_2 = \ldots = x_k$.
\end{enumerate}
\end{table}

Apart from that, as already mentioned, SMT solvers rarely work with constraints
of just one type and often employ a combination of various theories in the
Nelson-Oppen framework \cite{bib:nelson_oppen}. However, one of the requirements
of this framework is that all the theory solvers must share the implied
equalities. Therefore it is important to have an efficient algorithm also for
generating equalities that are implied by the given set of difference
constraints. Here we will briefly cover the one presented in \cite{bib:ms_dl}.
Of course we assume here that the set of constraint is satisfiable (i.e.
Bellman-Ford algorithm returned true for the generated graph). A naive way to
generate all the equalities is to perform the transitive closure of all the
constraints and check if we can find $x \leq y$ and $y \leq x$ for some $x$ and
$y$. However, this is quite expensive --- in the worst case would require
$\BigO(n^3)$. A more efficient algorithm is presented in Table \ref{tab:eq_gen}.
However, before we discuss it we need to introduce the concept of slack, i.e.
for every edge $(u, v)$ its slack is defined as
\[
sl(u, v) = \delta(v) - \delta(u) + w(u, v)
\]

The algorithm is based on the idea that if we consider only the edges have slack
equal to zero then any cycle formed by these edges will have weight equal to
zero. So when we have two vertices $u$ and $v$ in the same SCC (i.e. there is a
path from $u$ to $v$ and the other way around) and they have the same value
associated by $\delta$ then we have that the weight of both paths is equal to
zero. Which corresponds to $v - u \leq 0$ and $u - v \leq 0$, i.e. $u = v$.



%
% UTVPI
%

\newpage
\section{UTVPI}
\subsection{Introduction and comparison with DL}

UTVPI stands for Unit Two Variable Per Inequality and is also referred to as
Octagons \cite{bib:octagons}. Here we will be interested mostly in the question
of satisfiability of a set of constraints and not in the details of the
constraints as an abstract domain. In general the UTVPI constraints are of the
form
\[
a x + b y \leq c
\]
where $a, b \in \{ -1, 0, 1 \}$ and $c$ is either $\Rationals$ or $\Integers$.
UTVPI constraints are quite interesting because they are a bit more expressive
than difference logic and so allow more problems to be encoded, yet there still
exist efficient polynomial decision procedures for them. Furthermore in case of
$\Integers$ relaxing any requirement of the UTVPI (allowing non-unit
coefficients or more variables) leads to NP-complete decision procedures.

The first decision procedure for UTVPI was presented in \cite{bib:beyond_finite}
and showed that transitive closure of the set of constraints with respect to the
transitive and in case of $\Integers$ also tightening rules (Table
\ref{tab:inference}) is unsatisfiable if and only if it contains a
contradiction, like $0 \leq d$ where $d < 0$.

Note the importance of the tightening rule for $\Integers$. The additional
difficulty of $\Integers$ constraints is that they can have solutions that are
not in $\Integers$. So they might be satisfiable in $\Rationals$ but not in
$\Integers$. The tightening rule basically provides additional information that
if we have a constraint that is equivalent to $2x \leq d$ and $d$ is odd then
we can tighten the bound on $x$ (i.e. we have $2x \leq d - 1$).

\begin{table}[b]
\caption{Inference rules for UTVPI}
\label{tab:inference}
\centering
\begin{align}
& \frac{ ax + by \leq c  \qquad  -ax + b'z \leq d }
     { by +b'z \leq c + d}
\tag{Transitive} \\
& \notag \\
& \frac{ ax + by \leq c  \qquad  ax - by \leq d }
     { ax \leq \lfloor (c + d) / 2 \rfloor }
\tag{Tightening}
\end{align}
\end{table}
%
Also note that we will not consider here the case of strict inequalities ---
they can be handled as in case of DL. Instead we will focus more on the case
when the domain is $\Integers$, because for UTVPI this case is actually
more difficult than the one with $\Rationals$.

\subsection{Algorithms}

The procedure presented in \cite{bib:beyond_finite} has been improved by
\cite{bib:harvey_stuckey} but the worst case time complexity remained the same
--- $\BigO(m n^2)$. However, a new approach was proposed in \cite{bib:ms_utvpi}
which does not involve the transitive closure and uses a graph representation of
the constraints to solve satisfiability (note the similarity to the difference
logic). The complexity of the procedure is equal to the complexity of negative
cycle detection algorithm. So using Bellman-Ford algorithm it is possible to
achieve $\BigO(m n)$. One of the main ideas behind the approach is to construct
a graph with two vertices for each variable.

For every variable $x$ in the set of constraints we have two vertices $x^-$ and
$x^+$ that represent $-x$ and $x$. We will sometimes use $-v$ to denote $x^-$ if
$v$ represents $x^+$ and the other way around. Now for every constraint we add
one or two edges, as presented in the Table \ref{tab:utvpi_edges}.

\begin{table}
\caption{Edges in the UTVPI graph}
\label{tab:utvpi_edges}
\[
\begin{array}{c|c|c}
\textrm{UTVPI constraint} & \textrm{DL constraints} & \textrm{Graph Edges} \\
\hline \hline
x - y \leq c &
  x^+ - y^+ \leq c        \qquad y^- - x^- \leq c &
  y^+ \xrightarrow{c} x^+ \qquad x^- \xrightarrow{c} y^- \\
\hline
x + y \leq c &
  x^+ - y^- \leq c        \qquad y^+ - x^- \leq c &
  y^- \xrightarrow{c} x^+ \qquad x^- \xrightarrow{c} y^+ \\
\hline
- x - y \leq c &
  x^- - y^+ \leq c        \qquad y^- - x^+ \leq c &
  y^+ \xrightarrow{c} x^- \qquad x^+ \xrightarrow{c} y^- \\
\hline
x \leq c &
  x^+ - x^- \leq c &
  x^- \xrightarrow{2c} x^+ \\
\hline
-x \leq c &
  x^- - x^+ \leq c &
  x^+ \xrightarrow{2c} x^- \\
\end{array}
\]
\end{table}

There are a few interesting properties of such a graph. First of all the graph
is very similar to the one from decision procedure for DL. Yet it does not need
a special vertex $v_0$ that was is needed it in the constraint graphs for
difference logic to handle constraints of the from $x \leq c$. Another important
observation about this new graph is that whenever there is an edge $(u, v)$
there also is an edge $(-v, -u)$. Moreover this can be extended to paths ---
whenever we have a path from $u$ to $v$, there is a path from $-v$ to $-u$.
Apart from that, let $\delta$ be a valuation of all the vertices (i.e. a possible
solution to the constraints), then whenever we have a shortest path $SP(u, v)$ between
two vertices $u$ and $v$, we know that $\delta(v) - \delta(u) \leq wSP(u,
v)$ (where $wSP$ stands for $w \circ SP$, that is the weight of the shortest
path). Finally, one of the main result with respect to UTVPI/Octagons is the
following \cite{bib:octagons}.
%
\begin{lemma}
A set of UTVPI constraints is unsatisfiable in $\Rationals$ if and only if the
corresponding constraint graph contains a negative weight cycle.
\end{lemma}
%
Note that this is only true for $\Rationals$ and not for $\Integers$. As already
mentioned constraints over integers pose additional difficulties since they can
have a non-integer solutions. Therefore for the case of integers we need to
introduce the concept of tightening edges. They are defined as follows:
\[
T = \{ (u, -u) \mid wSP(u, -u) \text{ is odd} \}
\]
Note that this corresponds to an either $u^+ - u^- \leq c$ or $u^- - u^+ \leq
c$, where $c$ is odd. We are interested only in those edges because these are the
only ones that can cause the Tightening rule to apply. Therefore for each edge
in $T$ we define its weight as follows
\[
w_T(u, -u) = wSP(u, -u) - 1
\]
Let us denote the graph with those additional edges as $G_{\phi \cup T}$. One of
the main contributions of \cite{bib:ms_utvpi} is the lemma below.
\begin{lemma}
A set of UTVPI constraint $\phi$ is unsatisfiable in $\Integers$ if and only if
the graph $G_{\phi \cup T}$ has a negative weight cycle.
\end{lemma}
The na\"{i}ve approach would be to actually find and add all the additional
edges and run the negative cycle detection algorithm on this new graph. This
could be achieved by using either Floyd-Warshall or Johnson's All Pairs Shortest
Paths algorithms to find the tightening edges and then use negative cycle
detection algorithm. However, \cite{bib:ms_utvpi} presents a bit smarter
approach. It is summarized in Tabel \ref{tab:utvpi_sat}.

\begin{table}
\caption{Algorithm for satisfiability of UTVPI constraints}
\label{tab:utvpi_sat}
\begin{enumerate}
\item Construct $G_\phi$ graph.
\item Run a negative cycle detection algorithm on the graph.
  \begin{enumerate}
  \item If there is a negative cycle, return UNSAT.
  \item Otherwise the algorithm will return shortest paths from the source
    vertex $v_s$ to all other vertices. This can be treated as a feasible solution
    to the constraints. Note that $\delta(v) - \delta(u) \leq w(u, v)$ for all
    edges $(u, v)$.
  \item Let $E'$ be the set of edges from $G$ such that $(u, v) \in E'$ iff
    $\delta(u) - \delta(v) = w(u, v)$.
  \item Create graph $G'_\phi$ induced by $E'$
  \item Group the vertices into strongly connected components (for instance
    using Tarjan's algorithm \cite{bib:tarjan_dfs}, which is
    $\BigO(|V| + |E|)$).
  \item For each vertex $u \in V$, if $-u$ is in the same SCC as $u$ and
    $\delta(u) - \delta(-u)$ is odd then return UNSAT.
  \item Return SAT.
  \end{enumerate}
\end{enumerate}
\end{table}
The intuition behind the second part of the algorithm (i.e. SCC computation) is
to identify zero weight cycles and check if we can tighten some edge. If so then
the zero cycle in $G_\phi$ will result in a negative one in $G_{\phi \cup T}$ (i.e.
after adding the tightening edge). Moreover whenever there is a path $P$ between
$u$ and $v$ such that all the edges along the path have slack equal to zero,
then $wSP(u, v) = \delta(v) - \delta(u)$. Now if $u$ and $-u$ are in the same
SCC, then there is a path from $u$ to $-u$ and from $-u$ to $u$.
Moreover
\begin{align*}
\delta(u) - \delta(-u) & = wSP(-u, u) \\
\delta(-u) - \delta(u) & = wSP(u, -u) \\
0 & = wSP(-u, u) + wSP(u, -u) \\
\end{align*}
So $u$ and $-u$ form a zero weight cycle. If one of the paths is odd, then we
know that a corresponding tightening edge will be in $T$ and so from a negative
cycle. For more precise description and the formal soundness and completeness
proofs we refer to \cite{bib:ms_utvpi}.

The solver to be useful in DPLL(T) should produce theory conflicts, in this case
which inequalities cause a negative cycle. Then the solver can assert the theory
axiom saying that at least one of those equalities must be false. In order to
find the inequalities, we must track the reason why a given edge was added to
the graph. Then once we detect a negative cycle using e.g. Bellman-Ford
algorithm we can simply use the parent pointers to find the cycle. If the
unsatisfiability was detected in the second part of the algorithm, then all we
need to do is to find the path from $u$ to $-u$ and back in the induced graph
(this can be done by Breadth First Search) --- again the cycle is a proof that
there are no solutions in $\Integers$.

Finally, as in the case of DL, in order for the theory solver to be used in the
DPLL(T) framework, should be capable of generating all equalities that are
consequences of the considered inequalities. We will cover here two possible
approaches, as introduced in \cite{bib:ms_utvpi}. The first, quite simple way of
generating all inequalities is presented in Table~\ref{tab:naive_eq}. The idea
behind the algorithm is pretty much the same as for the one for difference
logic. That is if we know that
$\delta_{\phi \cup T}(x^+) = \delta_{\phi \cup T}(y^+)$
and both vertices are in a cycle of weight zero, then they are equal.

\begin{table}
\caption{Simple algorithm for equality generation}
\label{tab:naive_eq}
\begin{enumerate}
\item Construct $G_{\phi \cup T}$ graph and run a negative cycle detection
  algorithm to calculate a feasible solution $\delta_{\phi \cup T}$.
\item Let $E_0$ be a set of edges such that $e \in E_0$ iff $sl(e) = 0$.
\item Create the subgraph $G_0$ induced by the $E_0$ set and group the vertices
  into SCCs.
\item If vertices $x^+$ and $y^+$ are in the same SCC and
  $\delta_{\phi \cup T}(x^+) = \delta_{\phi \cup T}(y^+)$, then report that
  $x = y$.
\end{enumerate}
\end{table}

\begin{table}
\caption{Improved algorithm for equality generation}
\label{tab:fast_eq}
\begin{enumerate}
\item Construct graph $G_\phi$ and run the satisfiability algorithm on it. We
  assume that the algorithm returned SAT and a possible solution $\delta$
\item Calculate $E_2 = \{ (u, v) \mid sl(u, v) \leq 2 \}$ and the graph $G_2$
  induced by $E_2$.
\item Let $T_2$ be the set of tightening edges (initially set to $\emptyset$)
  that we are interested in.
\item For each vertex $v$
  \begin{enumerate}
  \item Find the path $P_v$ in $G_2$ from $v$ to $-v$ with the smallest slack (if
    any). This can be achieved using breadth-first search or Dijkstra's
    algorithm.
  \item If $P_v$ exists
    \begin{enumerate}
    \item Let $wSP(v, -v) = \delta(-v) - \delta(v) + sl(P_v)$.
    \item If $wSP(v, -v)$ is odd then add the edge $(v, -v)$ to $T_2$ and assign
      it weight $w_{T_2}(v, -v) = wSP(v, -v) - 1$.
    \end{enumerate}
  \end{enumerate}
\item Add all the edge in $T_2$ to the $G_\phi$ graph to get
  $G_{\phi \cup T_2}$.
\item Proceed as in the previous algorithm but with
  $G_{\phi \cup T_2}$ instead of $G_{\phi \cup T}$.
\end{enumerate}
\end{table}
However, this way of calculating the equalities requires us to compute all the
tightening edges $T$ and then add them to the graph. As before an improved
algorithm would avoid finding all those edges and use only the ones that can
lead to a cycle of weight zero. The improved algorithm makes use of an
intermediate lemma that is presented below \cite{bib:ms_utvpi}.

\begin{lemma}
Assuming $G_{\phi \cup T}$ has no negative cycles and if $C$ is a zero weight
cycle in $G_{\phi \cup T}$ containing a tightening edge $(u, -u)$ then there is
a cycle $C'$ in $G_\phi$ containing $u$ and $-u$ such that $w(C') \leq 2$.
\end{lemma}

This builds on another property that whenever there is a cycle $C$ in $G_{\phi \cup
T}$ then there is cycle $C'$ with at most two tightening edges, such that $w(C')
< 0$ or $w(C') < w(C)$. This gives us the ability to consider, without loss of
generality, cycles with at most two tightening edges.

Now the idea behind the finding only the tightening edges that are useful for
equality generation is that if the cycle $C$ has at most two tightening edges
then there is a cycle $C'$ without those edges and its weight is not greater
than two (i.e. each of the tightening edges can decrease the weight by one only,
so if we remove them the weight can get larger by at most two).
The improved algorithm that taking advantage of this is presented in the Table
\ref{tab:fast_eq}. It only really changes the first step of the previous one ---
instead of calculating the whole $G_{\phi \cup T}$, it calculates only those
tightening edges that are useful, adds them to the $G_\phi$ graph and then
proceeds as before.


\newpage
\section{Implementation}
\subsection{Introduction}

The accompanying UTPVI theory solver for Z3 SMT Solver \cite{bib:z3} is
implemented in around 1500 lines of C++ (taking advantage of recent C++11
standard \cite{bib:cc11}). It uses Boost Graph Library
\cite{bib:boost,bib:boost_graph} for graph manipulation and GMP \cite{bib:gmp}
for arbitrary precision arithmetic. Due to time constraints it is mainly a
proof-of-concept and does not have all the features necessary for usage in a
combination with other theories. Currently it implements:
\begin{itemize}
\item decision procedure for UTVPI (and thus SVPI) for integers, both using
  basic types of C++ like int, as well as arbitrary precision integers from GMP
  library
\item decision procedure for UTVPI (and thus SVPI) for rationals using arbitrary
  precision rationals from GMP library\footnote{In fact current implementation
  should work with basic types like \texttt{float} or \texttt{double}, however,
  due to their imprecision, this would probably not be very useful.}
\item generation of theory conflicts
\item backtracking
\item integration with Z3
\end{itemize}
Initially the implementation was to be based on the one for Fx7 SMT solver
\cite{bib:fx7}. However, it implements an older algorithm described in
\cite{bib:harvey_stuckey}. Therefore this implementation is not based on the
procedure from Fx7 and implements the more recent algorithms from
\cite{bib:ms_utvpi} instead.

One of the main highlights of this implementation is the ability to easily
instantiate solvers for rationals/integers with various representations. For
instance the following snippet creates theories for checking satisfiability for
integers that fit in \texttt{int}, integers of arbitrary precision and rationals
of arbitrary precision respectively.
\begin{lstlisting}
Z3_theory mach_integers  = MkTheory<UtvpiGraphZ, int>(context);
Z3_theory arbt_integers  = MkTheory<UtvpiGraphZ, mpz_class>(context);
Z3_theory arbt_rationals = MkTheory<UtvpiGraphQ, mpq_class>(context);
\end{lstlisting}
Where \texttt{context} is \texttt{Z3\_context}, \texttt{mpz\_class} and
\texttt{mpq\_class} are GMP's classes for arbitrary precision integers and
rationals respectively\footnote{GMP offers both a basic C as well as somewhat
more convenient C++ interface, which overloads basic arithmetic operators.}. Note
that this includes running all the described graph algorithms with weights of
the specified types, as well as using the appropriate functions from Z3 to parse
and create numerals (in case of \texttt{int} we use
\texttt{Z3\_get\_numeral\_int}, in case of \texttt{mpz\_class} we need to
go through a string and use
use \texttt{Z3\_get\_numeral\_string}).

The project uses Z3 parser and predefines two predicates \texttt{Utvpi} and
\texttt{Svpi} and introduces additional sort \texttt{Sign} with two constants
\texttt{Minus} and \texttt{Plus} that can be used as follows:
\begin{verbatim}
(declare-fun x () Int)
(declare-fun y () Int)

(assert (or (Utvpi Plus x Minus y (~ 1)) (Utvpi Plus x Minus y 1)))
(assert (Svpi Plus y 0))
(assert (Svpi Minus x (~ 1)))
\end{verbatim}
This translates to
\[
(x - y \leq -1 \vee x - y \leq 1 )
\land
y \leq 0
\land
- x \leq -1
\]
Of course this can be easily changed as desired.

The tool also outputs some information on how it is solving the problem. For the
above example the output is:
\begin{verbatim}
Parsing file: test.smt
Parsed the following formula:
(and (or (Utvpi Plus x Minus y (- 1)) (Utvpi Plus x Minus y 1))
     (Svpi Plus y 0)
     (Svpi Minus x (- 1)))
Z3: Assigned (Svpi Plus y 0) to 1
Z3: Assigned (Svpi Minus x -1) to 1
Z3: Push
Z3: Assigned (Utvpi Plus x Minus y -1) to 1
UtvpiGraphZ: Found negative weight cycle without tightening.
UtvpiGraph: Reason for negative cycle:
[..]
SatCheck: UNSAT.
SatCheck: asserting theory axiom:
(not (and (and (and (Svpi Minus x -1) (Utvpi Plus x Minus y -1))
               (Svpi Plus y 0))
          (Utvpi Plus x Minus y -1)))
Z3: Pop
Z3: Assigned (Utvpi Plus x Minus y 1) to 1
Z3: SAT
Z3: Reset
\end{verbatim}
Which clearly shows how everything works:
\begin{enumerate}
\item First Z3 assigns the two \texttt{SVPI} predicates the value true (they
  are asserted so they must be true for the formula to by satisfiable).
\item Then Z3 makes \texttt{Push} callback for the case-split caused by the
  asserted disjunction.
\item Assigning true for the first disjuncts causes a theory conflict from the
  UTVPI solver (clearly $y \leq 0 \land x \geq 1 \land x - y \leq -1$ is
  unsatisfiable).
\item Z3 reacts making the \texttt{Pop} callback and assigning true to the other
  disjunct.
\item Theory solver returns that this is satisfiable.
\item Thus the formula is satisfiable.
\end{enumerate}
Of course if we modify the formula removing the second disjunct, the result is:
\begin{verbatim}
Parsing file: test.smt
Parsed the following formula:
(and (Utvpi Plus x Minus y (- 1)) (Svpi Plus y 0) (Svpi Minus x (- 1)))
Z3: Assigned (Utvpi Plus x Minus y -1) to 1
Z3: Assigned (Svpi Plus y 0) to 1
Z3: Assigned (Svpi Minus x -1) to 1
UtvpiGraphZ: Found negative weight cycle without tightening.
UtvpiGraph: Reason for negative cycle:
[..]
SatCheck: UNSAT.
SatCheck: asserting theory axiom:
(not (and (and (and (Utvpi Plus x Minus y -1) (Svpi Plus y 0))
               (Utvpi Plus x Minus y -1))
          (Svpi Minus x -1)))
Z3: UNSAT
Z3: Reset
\end{verbatim}


%
% Conclusions
%

\newpage
\section{Conclusions}

Both Difference Logic and UTVPI seem to be a great compromise for satisfiability
checking in the context of modern SMT solvers. They both allow very efficient
graph-based algorithms for problems such as satisfiability or equality
generation. And at the same time they often naturally arise in software
verification (array bounds checking, etc.) or in other problems like job
scheduling.

This report provided a brief survey of the current state-of-the-art in the
satisfiability checking for Difference Logic and UTVPI. It focused on providing
intuition about various algorithms and reasons of their efficiency. Apart from
that a proof-of-concept implementation of UTVPI theory solver for the Z3 SMT
solver has been developed.


\bibliographystyle{plain}
\bibliography{report}

\end{document}
